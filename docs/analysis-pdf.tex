% Options for packages loaded elsewhere
\PassOptionsToPackage{unicode}{hyperref}
\PassOptionsToPackage{hyphens}{url}
%
\documentclass[
]{article}
\usepackage{lmodern}
\usepackage{amssymb,amsmath}
\usepackage{ifxetex,ifluatex}
\ifnum 0\ifxetex 1\fi\ifluatex 1\fi=0 % if pdftex
  \usepackage[T1]{fontenc}
  \usepackage[utf8]{inputenc}
  \usepackage{textcomp} % provide euro and other symbols
\else % if luatex or xetex
  \usepackage{unicode-math}
  \defaultfontfeatures{Scale=MatchLowercase}
  \defaultfontfeatures[\rmfamily]{Ligatures=TeX,Scale=1}
\fi
% Use upquote if available, for straight quotes in verbatim environments
\IfFileExists{upquote.sty}{\usepackage{upquote}}{}
\IfFileExists{microtype.sty}{% use microtype if available
  \usepackage[]{microtype}
  \UseMicrotypeSet[protrusion]{basicmath} % disable protrusion for tt fonts
}{}
\makeatletter
\@ifundefined{KOMAClassName}{% if non-KOMA class
  \IfFileExists{parskip.sty}{%
    \usepackage{parskip}
  }{% else
    \setlength{\parindent}{0pt}
    \setlength{\parskip}{6pt plus 2pt minus 1pt}}
}{% if KOMA class
  \KOMAoptions{parskip=half}}
\makeatother
\usepackage{xcolor}
\IfFileExists{xurl.sty}{\usepackage{xurl}}{} % add URL line breaks if available
\IfFileExists{bookmark.sty}{\usepackage{bookmark}}{\usepackage{hyperref}}
\hypersetup{
  pdftitle={Laptop Pricing},
  pdfauthor={Attila Szuts},
  hidelinks,
  pdfcreator={LaTeX via pandoc}}
\urlstyle{same} % disable monospaced font for URLs
\usepackage[margin=1in]{geometry}
\usepackage{color}
\usepackage{fancyvrb}
\newcommand{\VerbBar}{|}
\newcommand{\VERB}{\Verb[commandchars=\\\{\}]}
\DefineVerbatimEnvironment{Highlighting}{Verbatim}{commandchars=\\\{\}}
% Add ',fontsize=\small' for more characters per line
\usepackage{framed}
\definecolor{shadecolor}{RGB}{248,248,248}
\newenvironment{Shaded}{\begin{snugshade}}{\end{snugshade}}
\newcommand{\AlertTok}[1]{\textcolor[rgb]{0.94,0.16,0.16}{#1}}
\newcommand{\AnnotationTok}[1]{\textcolor[rgb]{0.56,0.35,0.01}{\textbf{\textit{#1}}}}
\newcommand{\AttributeTok}[1]{\textcolor[rgb]{0.77,0.63,0.00}{#1}}
\newcommand{\BaseNTok}[1]{\textcolor[rgb]{0.00,0.00,0.81}{#1}}
\newcommand{\BuiltInTok}[1]{#1}
\newcommand{\CharTok}[1]{\textcolor[rgb]{0.31,0.60,0.02}{#1}}
\newcommand{\CommentTok}[1]{\textcolor[rgb]{0.56,0.35,0.01}{\textit{#1}}}
\newcommand{\CommentVarTok}[1]{\textcolor[rgb]{0.56,0.35,0.01}{\textbf{\textit{#1}}}}
\newcommand{\ConstantTok}[1]{\textcolor[rgb]{0.00,0.00,0.00}{#1}}
\newcommand{\ControlFlowTok}[1]{\textcolor[rgb]{0.13,0.29,0.53}{\textbf{#1}}}
\newcommand{\DataTypeTok}[1]{\textcolor[rgb]{0.13,0.29,0.53}{#1}}
\newcommand{\DecValTok}[1]{\textcolor[rgb]{0.00,0.00,0.81}{#1}}
\newcommand{\DocumentationTok}[1]{\textcolor[rgb]{0.56,0.35,0.01}{\textbf{\textit{#1}}}}
\newcommand{\ErrorTok}[1]{\textcolor[rgb]{0.64,0.00,0.00}{\textbf{#1}}}
\newcommand{\ExtensionTok}[1]{#1}
\newcommand{\FloatTok}[1]{\textcolor[rgb]{0.00,0.00,0.81}{#1}}
\newcommand{\FunctionTok}[1]{\textcolor[rgb]{0.00,0.00,0.00}{#1}}
\newcommand{\ImportTok}[1]{#1}
\newcommand{\InformationTok}[1]{\textcolor[rgb]{0.56,0.35,0.01}{\textbf{\textit{#1}}}}
\newcommand{\KeywordTok}[1]{\textcolor[rgb]{0.13,0.29,0.53}{\textbf{#1}}}
\newcommand{\NormalTok}[1]{#1}
\newcommand{\OperatorTok}[1]{\textcolor[rgb]{0.81,0.36,0.00}{\textbf{#1}}}
\newcommand{\OtherTok}[1]{\textcolor[rgb]{0.56,0.35,0.01}{#1}}
\newcommand{\PreprocessorTok}[1]{\textcolor[rgb]{0.56,0.35,0.01}{\textit{#1}}}
\newcommand{\RegionMarkerTok}[1]{#1}
\newcommand{\SpecialCharTok}[1]{\textcolor[rgb]{0.00,0.00,0.00}{#1}}
\newcommand{\SpecialStringTok}[1]{\textcolor[rgb]{0.31,0.60,0.02}{#1}}
\newcommand{\StringTok}[1]{\textcolor[rgb]{0.31,0.60,0.02}{#1}}
\newcommand{\VariableTok}[1]{\textcolor[rgb]{0.00,0.00,0.00}{#1}}
\newcommand{\VerbatimStringTok}[1]{\textcolor[rgb]{0.31,0.60,0.02}{#1}}
\newcommand{\WarningTok}[1]{\textcolor[rgb]{0.56,0.35,0.01}{\textbf{\textit{#1}}}}
\usepackage{graphicx,grffile}
\makeatletter
\def\maxwidth{\ifdim\Gin@nat@width>\linewidth\linewidth\else\Gin@nat@width\fi}
\def\maxheight{\ifdim\Gin@nat@height>\textheight\textheight\else\Gin@nat@height\fi}
\makeatother
% Scale images if necessary, so that they will not overflow the page
% margins by default, and it is still possible to overwrite the defaults
% using explicit options in \includegraphics[width, height, ...]{}
\setkeys{Gin}{width=\maxwidth,height=\maxheight,keepaspectratio}
% Set default figure placement to htbp
\makeatletter
\def\fps@figure{htbp}
\makeatother
\setlength{\emergencystretch}{3em} % prevent overfull lines
\providecommand{\tightlist}{%
  \setlength{\itemsep}{0pt}\setlength{\parskip}{0pt}}
\setcounter{secnumdepth}{-\maxdimen} % remove section numbering

\title{Laptop Pricing}
\author{Attila Szuts}
\date{02/01/2021}

\begin{document}
\maketitle
\begin{abstract}
This analysis will try to answer how you can price a laptop based on its
specifications and how you can try to find good deals among them. It
uses 1300 data points to build a linear regression model on price using
different properties like company that manufactured it, cpu model, RAM
size, etc.
\end{abstract}

{
\setcounter{tocdepth}{2}
\tableofcontents
}
\hypertarget{introduction}{%
\section{Introduction}\label{introduction}}

Following my interest of tech gadgets I wanted to investigate a
question, that was bothering me for a while. How can we accurately tell
and compare prices of different laptops? It is hard by itself to price
them, as usually they are priced at a premium than similar speced custom
built PCs, not only because of the portability but also because the
assembly costs. So in order to do this, I decided to build a linear
regression model for pricing laptops based on their specifications
(screen size, manufacturer, RAM, etc.)

\hypertarget{data-prep}{%
\section{Data prep}\label{data-prep}}

\hypertarget{data-collection}{%
\subsection{Data collection}\label{data-collection}}

Data was collected from
\href{https://www.kaggle.com/muhammetvarl/laptop-price/}{kaggle}, thus I
do not know the original source of it. Since this is similar to
administrative data, it is likely that the values are accurate and there
is no inherent classical measurement error present. However, there can
still be some abnormalities that can distort the model. For example it
might be very important from a pricing perspective how long ago since
the laptop has been released, since prices tend to decrease after the
initial hype. Also, it can happen quite easily that someone mistyped
something if data was entered manually. The question still remains,
which is that is price dependent exclusively on specs? Likely that no,
there are other factors at play as well, such as design, build quality,
materials used, marketing, etc. Also, there are some important
properties not present, like battery size, number of ports, keyboard
type, etc but for our purpose it will be a good enough approximation of
price.

To get as much information as possible, I will try to use all variables
in the model, and I'll only exclude them if I have to.

\hypertarget{data-cleaning}{%
\subsection{Data cleaning}\label{data-cleaning}}

I had to clean almost every variable, to clean numeric variables (RAM,
SSD size, HDD size, etc.). Alongside this, I also had to mine
information from different variables such as IPS, touchscreen properties
of screens, or CPU model, manufacturer, frequency, etc. I created
functions that could extract these informations into new ones.

I recoded the baseline for my variables. I choose these partly based on
my personal interest and also on sample size to consider the SE of the
coefficients.

\begin{itemize}
\tightlist
\item
  Operating system -\textgreater{} windows 10
\item
  Company -\textgreater{} Lenovo
\item
  Type -\textgreater{} Notebook
\item
  Screen size category -\textgreater{} screen\_mid
\item
  Screen resolution -\textgreater{} 1920x1080
\item
  CPU model -\textgreater{} core i7
\item
  Memory type -\textgreater{} ssd
\item
  GPU type -\textgreater{} integrated
\end{itemize}

Finally, I created an 80-20 train-test split.

\hypertarget{descriptives}{%
\subsection{Descriptives}\label{descriptives}}

We can see in the \protect\hyperlink{appendix}{appendix} that the log
transformed version of prices is the closest to distribution.

Detailed plots on the pattern of association between log-price and
covariates can be found in the \protect\hyperlink{appendix}{appendix}.

\begin{table}

\caption{\label{tab:unnamed-chunk-6}Summary statistics for price}
\centering
\begin{tabular}[t]{r|r|r|r|r|r}
\hline
Min. & 1st Qu. & Median & Mean & 3rd Qu. & Max.\\
\hline
174 & 598.9225 & 979 & 1117.583 & 1468.75 & 6099\\
\hline
\end{tabular}
\end{table}

\hypertarget{model-parameters}{%
\section{Model parameters}\label{model-parameters}}

For the baseline model I am going to use a simple regression of log
price on the manufacturer company. I decided to use ln prices, as it was
skewed to the left, having a long right tail and this transformation
made it normal. This model gives the average price percentage change for
each manufacturer compared to a baseline, which in this case is Lenovo.

reg1: ln\_price \textasciitilde{} company

To find out which coefficients provide the best fit for my data I ran a
simulation to find the best possible fit using AIC with the
\texttt{stepAIC} function.

However this model contained
\protect\hyperlink{multicollinearity}{multicollinearity}, so I had to
drop some variables. After this step I came up with the final form:

reg3: ln\_price \textasciitilde{} company + type\_name + inches + ram +
screen\_category + cpu\_model + memory\_type + ssd\_size + hdd\_size

\hypertarget{model-interpretation}{%
\subsection{Model interpretation}\label{model-interpretation}}

This model can be interpreted as giving the price percentage change,
when we change a property. \(exp(\alpha)\) gives us the average price
for a Lenovo Notebook with Windows 10, Intel core i7 CPU, medium sized
FullHD display, SSD and integrated GPU which is 765.7421174 Euros. We
can see how the price changes, if we change some property of this
imaginary laptop or we can also analyse residuals to find the best deal
for a certain category. You can find the detailed comparison in the
\protect\hyperlink{detailed-model-comparison}{appendix}.

We can see that among the companies, Lenovo is somewhere in the lower
end having quite a few more expensive counterparts (e.g.~Apple), but
also having more budget options (e.g.~Acer).

Ultrabooks, Gaming, Convertible and workstation products are all
significantly more expensive compared to Notebooks. Netbooks are
cheaper, but marginally and insignificantly.

Inches, that is screen size is not a significant predictor, however, the
categories were significant. Both smaller and larger screens are around
20\% more expensive compared to medium sized screens. This is probably
because of other factors (like premium devices being smaller so they are
more portable or gaming pcs having large screens).

1GB more RAM will cost you 2\% more on average.

The Intel core i7 is a very expensive cpu, and virtually any other
option will cost you less (sometimes even as much as 70\%!). This is a
very interesting finding, since the CPU is basically the heart of the
machine, if you can find a suitable cpu for your needs, then you can
potentially save a lot on a pc. But in this paper I am not going to go
into more details on this for obvious reasons.

Having both SSD and HDD will be 23\% higher on average compared to
having only SSD. However, interestingly 1 GB more storage is associated
with just a fraction of a percent higher prices on average. But this can
still be an impactful coefficient if we investigate the standardized
betas.

If you are interested which properties of a laptop are the most
important, you can take a look at the
\protect\hyperlink{standardized-residuals}{appendix} where I dive into
more detail on the coefficients, and investigate the standardized
version of them which can tell us how much they impact the slope of the
regression compared to each other. As I suspected earlier, the most
important of them are RAM size, CPU type, SSD and HDD size and finally
screen size category.

Our model has a very high \(R^2=0.85\) compared to the baseline
\(R^2=0.16\). I think, this is a relatively accurate and robust result.
We will check this with test sample.

\hypertarget{residual-analysis}{%
\section{Residual analysis}\label{residual-analysis}}

Now that we are sure our model met all our assumptions (that can be
checked in the \protect\hyperlink{model-assumptions}{appendix} we can
analyse the residuals to find the most valuable deals. I limited the
results to be under €2000 because above that it might be that our model
is less accurate and there are other factors in pricing premium products
as well. Based on this, we get the following results.

\begin{table}

\caption{\label{tab:unnamed-chunk-12}Underpriced laptops}
\centering
\begin{tabular}[t]{l|l|r|r}
\hline
Company & Product Name & Price - Actual & Price - Predicted\\
\hline
Asus & G701VO-IH74K (i7-6820HK/32GB/2x & 1279.0 & 2365.0639\\
\hline
HP & 250 G6 & 393.9 & 718.2363\\
\hline
HP & 15-BS101nv (i7-8550U/8GB/256GB/FHD/W10) & 659.0 & 1189.7375\\
\hline
Lenovo & IdeaPad 510s-14IKB & 799.0 & 1417.9101\\
\hline
Lenovo & Yoga 500-14ISK & 638.0 & 1115.8852\\
\hline
Lenovo & Yoga 500-14ISK & 638.0 & 1115.8852\\
\hline
\end{tabular}
\end{table}

\hypertarget{robustness-checks}{%
\section{Robustness checks}\label{robustness-checks}}

Let us see now the test sample results, how well does the model predict
laptop prices.

You can see in the appendix the detailed comparison, but for the most
part, all coefficients are the same (except for cases where the group
size is small). The \(R^2=0.86\) is also very similar to the one we got
with the training sample. So we can say it with confidence that there is
some reality to our model, if this sample is representative. The
\(\text{Y} - \hat{\text{Y}}\) plot is showing a good fit as well.

\includegraphics{analysis-pdf_files/figure-latex/unnamed-chunk-14-1.pdf}

One more thing that I find worth mentioning is the case of overfitting.
This might be the case, since we have a lot of different variables,
however it is very hard to say anything about this. Hopefully thinking
about external validity and finding different datasets that could be
tested with this model could help shed some light on this question.

One of the most useful things to do with this model is to test it with
prices from a different time. Maybe there we could uncover different
patterns of associations for different laptops. For example, low-end
pc-s price might drop significantly after release, but their performance
could drop even faster so getting a deal on them might not be worth it.
Similarly, high-end gaming laptops can hold their prices for a longer
period of time and dropping in relative performance but not as much, so
it would be worth it to buy it. It would also be interesting to see and
try to fit this model on PCs or Smartphones as they share a lot of
similiar properties.

\hypertarget{summary}{%
\section{Summary}\label{summary}}

Finally, let me wrap up the findings here. All in all, the model I built
fits the data very well and is suitable to find underpriced laptops.
\(Exp(\alpha)\) gives us the average price for a Lenovo Notebook with
Windows 10, Intel core i7 CPU, medium sized FullHD display, SSD and
integrated GPU which is 765.74 Euros. The most influential coefficients
in the model were RAM size, CPU type, SSD and HDD size and finally,
screen size category.

To make it more useful, and develop this project further, it would be a
great next step, to investigate the question of external validity. It
would be great if I could find data on different properties (for example
benchmark scores on performance for each model) and try to model price
with those or build a competing model with this and compare them. Or it
would be another interesting research topic to try and model the price
conditioned on the number of days since release. To investigate external
validity even further one could try to apply this model on a dataset
with PCs and Smartphones.

\hypertarget{appendix}{%
\section{Appendix}\label{appendix}}

\hypertarget{log-price-distribution}{%
\subsection{Log Price distribution}\label{log-price-distribution}}

\includegraphics{analysis-pdf_files/figure-latex/unnamed-chunk-15-1.pdf}
\includegraphics{analysis-pdf_files/figure-latex/unnamed-chunk-15-2.pdf}
\includegraphics{analysis-pdf_files/figure-latex/unnamed-chunk-15-3.pdf}

\hypertarget{pattern-of-association-between-ln_price-and-predictors}{%
\subsection{Pattern of association between ln\_price and
predictors}\label{pattern-of-association-between-ln_price-and-predictors}}

\begin{Shaded}
\begin{Highlighting}[]
\KeywordTok{boxfun}\NormalTok{(laptop}\OperatorTok{$}\NormalTok{company)}
\end{Highlighting}
\end{Shaded}

\includegraphics{analysis-pdf_files/figure-latex/unnamed-chunk-16-1.pdf}

\begin{Shaded}
\begin{Highlighting}[]
\KeywordTok{boxfun}\NormalTok{(laptop}\OperatorTok{$}\NormalTok{type_name)}
\end{Highlighting}
\end{Shaded}

\includegraphics{analysis-pdf_files/figure-latex/unnamed-chunk-16-2.pdf}

\begin{Shaded}
\begin{Highlighting}[]
\KeywordTok{scatterfun}\NormalTok{(laptop}\OperatorTok{$}\NormalTok{inches)}
\end{Highlighting}
\end{Shaded}

\includegraphics{analysis-pdf_files/figure-latex/unnamed-chunk-16-3.pdf}

\begin{Shaded}
\begin{Highlighting}[]
\KeywordTok{boxfun}\NormalTok{(laptop}\OperatorTok{$}\NormalTok{screen_category)}
\end{Highlighting}
\end{Shaded}

\includegraphics{analysis-pdf_files/figure-latex/unnamed-chunk-16-4.pdf}

\begin{Shaded}
\begin{Highlighting}[]
\KeywordTok{boxfun}\NormalTok{(laptop}\OperatorTok{$}\NormalTok{touchscreen)}
\end{Highlighting}
\end{Shaded}

\includegraphics{analysis-pdf_files/figure-latex/unnamed-chunk-16-5.pdf}

\begin{Shaded}
\begin{Highlighting}[]
\KeywordTok{boxfun}\NormalTok{(laptop}\OperatorTok{$}\NormalTok{ips)}
\end{Highlighting}
\end{Shaded}

\includegraphics{analysis-pdf_files/figure-latex/unnamed-chunk-16-6.pdf}

\begin{Shaded}
\begin{Highlighting}[]
\KeywordTok{boxfun}\NormalTok{(laptop}\OperatorTok{$}\NormalTok{resolution)}
\end{Highlighting}
\end{Shaded}

\includegraphics{analysis-pdf_files/figure-latex/unnamed-chunk-16-7.pdf}

\begin{Shaded}
\begin{Highlighting}[]
\KeywordTok{scatterfun}\NormalTok{(laptop}\OperatorTok{$}\NormalTok{ram)}
\end{Highlighting}
\end{Shaded}

\includegraphics{analysis-pdf_files/figure-latex/unnamed-chunk-16-8.pdf}

\begin{Shaded}
\begin{Highlighting}[]
\KeywordTok{scatterfun}\NormalTok{(laptop}\OperatorTok{$}\NormalTok{gpu_type)}
\end{Highlighting}
\end{Shaded}

\includegraphics{analysis-pdf_files/figure-latex/unnamed-chunk-16-9.pdf}

\begin{Shaded}
\begin{Highlighting}[]
\KeywordTok{boxfun}\NormalTok{(laptop}\OperatorTok{$}\NormalTok{cpu_manufac)}
\end{Highlighting}
\end{Shaded}

\includegraphics{analysis-pdf_files/figure-latex/unnamed-chunk-16-10.pdf}

\begin{Shaded}
\begin{Highlighting}[]
\KeywordTok{boxfun}\NormalTok{(laptop}\OperatorTok{$}\NormalTok{cpu_model)}
\end{Highlighting}
\end{Shaded}

\includegraphics{analysis-pdf_files/figure-latex/unnamed-chunk-16-11.pdf}

\begin{Shaded}
\begin{Highlighting}[]
\KeywordTok{scatterfun}\NormalTok{(laptop}\OperatorTok{$}\NormalTok{cpu_freq)}
\end{Highlighting}
\end{Shaded}

\includegraphics{analysis-pdf_files/figure-latex/unnamed-chunk-16-12.pdf}

\begin{Shaded}
\begin{Highlighting}[]
\KeywordTok{boxfun}\NormalTok{(laptop}\OperatorTok{$}\NormalTok{memory_type)}
\end{Highlighting}
\end{Shaded}

\includegraphics{analysis-pdf_files/figure-latex/unnamed-chunk-16-13.pdf}

\begin{Shaded}
\begin{Highlighting}[]
\KeywordTok{scatterfun}\NormalTok{(laptop}\OperatorTok{$}\NormalTok{ssd_size)}
\end{Highlighting}
\end{Shaded}

\includegraphics{analysis-pdf_files/figure-latex/unnamed-chunk-16-14.pdf}

\begin{Shaded}
\begin{Highlighting}[]
\KeywordTok{scatterfun}\NormalTok{(laptop}\OperatorTok{$}\NormalTok{hdd_size)}
\end{Highlighting}
\end{Shaded}

\includegraphics{analysis-pdf_files/figure-latex/unnamed-chunk-16-15.pdf}

\begin{Shaded}
\begin{Highlighting}[]
\KeywordTok{boxfun}\NormalTok{(laptop}\OperatorTok{$}\NormalTok{op_sys)}
\end{Highlighting}
\end{Shaded}

\includegraphics{analysis-pdf_files/figure-latex/unnamed-chunk-16-16.pdf}

\begin{Shaded}
\begin{Highlighting}[]
\KeywordTok{scatterfun}\NormalTok{(laptop}\OperatorTok{$}\NormalTok{weight)}
\end{Highlighting}
\end{Shaded}

\includegraphics{analysis-pdf_files/figure-latex/unnamed-chunk-16-17.pdf}

\hypertarget{model-assumptions}{%
\subsection{Model Assumptions}\label{model-assumptions}}

To see if my model meets all the assumptions of multiple linear
regression I will investigate outliers and influential cases,
multicollinearity, residuals and the independence of errors.

\hypertarget{outliers-and-influential-cases}{%
\subsubsection{Outliers and influential
cases}\label{outliers-and-influential-cases}}

There are in total 52 outliers. This in itself is not necessarily a bad
thing, it might be that they are very good deals. However it is worth
taking a look at them to see if they exert undue influence on the model
which could in turn distort the results. Investigating the cook's
distance of the outliers we can find that there ar 0 observation above 1
which means they are not influential cases.

\hypertarget{multicollinearity}{%
\subsubsection{Multicollinearity}\label{multicollinearity}}

Upon inspecting the first model (reg3), I found that there is
multicollinearity among the variables. So I had to investigate which
variables cause this, and I found that there was multicollinearity among
the operating system, resolution, gpu type and cpu model confounders. I
experimented to see if there are some combination of these variables
that can be included in the model, without multicollinearity but only
the cpu model could be used. So due to this, I had to go back and
exclude the aforementioned variables. But even with this modification,
there still may be some multicollinearity among my variables, as the
average VIF is above 1.

Multicollinearity is an issue, since it limits the \(R^2\) of the model
and increases the SE of the \(\beta\) coefficients. This makes it more
difficult to interpret the results and the model parameters will vary a
lot based on the sample provided.

\hypertarget{residuals---homoskedasticity-and-normality}{%
\subsubsection{Residuals - homoskedasticity and
normality}\label{residuals---homoskedasticity-and-normality}}

\includegraphics{analysis-pdf_files/figure-latex/unnamed-chunk-19-1.pdf}

We can see on this plot, that the residuals are normally distributed,
because they follow a bell-shaped curve.

\includegraphics{analysis-pdf_files/figure-latex/unnamed-chunk-20-1.pdf}

And on this plot we can see, that there is no heteroskedasticity or
non-linearity in the data: points are scattered all over at random.

\hypertarget{independent-errors}{%
\subsubsection{Independent errors}\label{independent-errors}}

The Durbin-Watson test is 2.0345952272064, so the errors are independent
in the sample.

\hypertarget{detailed-model-comparison}{%
\subsection{Detailed model comparison}\label{detailed-model-comparison}}

\begin{table}
\begin{center}
\begin{tabular}{l c c}
\hline
 & Baseline - simple linear & Extended - Multiple regression \\
\hline
(Intercept)                   & $6.76$  & $6.64^{***}$  \\
                              & $$      & $(0.39)$      \\
companyAcer                   & $-0.44$ & $-0.09^{**}$  \\
                              & $$      & $(0.03)$      \\
companyApple                  & $0.53$  & $0.15^{*}$    \\
                              & $$      & $(0.07)$      \\
companyAsus                   & $0.05$  & $-0.04$       \\
                              & $$      & $(0.03)$      \\
companyChuwi                  & $-1.05$ & $0.22$        \\
                              & $$      & $(0.20)$      \\
companyDell                   & $0.17$  & $0.03$        \\
                              & $$      & $(0.02)$      \\
companyFujitsu                & $-0.18$ & $-0.11$       \\
                              & $$      & $(0.17)$      \\
companyGoogle                 & $0.64$  & $0.16$        \\
                              & $$      & $(0.15)$      \\
companyHP                     & $0.06$  & $0.07^{**}$   \\
                              & $$      & $(0.02)$      \\
companyHuawei                 & $0.50$  & $0.01$        \\
                              & $$      & $(0.18)$      \\
companyLG                     & $0.93$  & $0.43^{*}$    \\
                              & $$      & $(0.18)$      \\
companyMediacom               & $-1.10$ & $0.08$        \\
                              & $$      & $(0.16)$      \\
companyMicrosoft              & $0.57$  & $0.18$        \\
                              & $$      & $(0.10)$      \\
companyMSI                    & $0.62$  & $0.02$        \\
                              & $$      & $(0.05)$      \\
companyRazer                  & $1.17$  & $0.23^{*}$    \\
                              & $$      & $(0.10)$      \\
companySamsung                & $-0.03$ & $0.04$        \\
                              & $$      & $(0.14)$      \\
companyToshiba                & $0.30$  & $0.16^{***}$  \\
                              & $$      & $(0.04)$      \\
companyVero                   & $-1.41$ & $-0.06$       \\
                              & $$      & $(0.31)$      \\
companyXiaomi                 & $0.18$  & $-0.03$       \\
                              & $$      & $(0.14)$      \\
type\_name2 in 1 Convertible  &         & $0.10^{**}$   \\
                              &         & $(0.03)$      \\
type\_nameGaming              &         & $0.21^{***}$  \\
                              &         & $(0.03)$      \\
type\_nameNetbook             &         & $-0.04$       \\
                              &         & $(0.08)$      \\
type\_nameUltrabook           &         & $0.14^{***}$  \\
                              &         & $(0.03)$      \\
type\_nameWorkstation         &         & $0.66^{***}$  \\
                              &         & $(0.06)$      \\
inches                        &         & $-0.01$       \\
                              &         & $(0.02)$      \\
ram                           &         & $0.02^{***}$  \\
                              &         & $(0.00)$      \\
screen\_categoryscreen\_big   &         & $0.19^{***}$  \\
                              &         & $(0.05)$      \\
screen\_categoryscreen\_small &         & $0.21^{***}$  \\
                              &         & $(0.05)$      \\
cpu\_modela10-series          &         & $-0.38^{***}$ \\
                              &         & $(0.10)$      \\
cpu\_modela12-series          &         & $-0.32^{**}$  \\
                              &         & $(0.10)$      \\
cpu\_modela4-series           &         & $-0.88^{***}$ \\
                              &         & $(0.25)$      \\
cpu\_modela6-series           &         & $-0.78^{***}$ \\
                              &         & $(0.10)$      \\
cpu\_modela8-series           &         & $-0.68^{**}$  \\
                              &         & $(0.25)$      \\
cpu\_modela9-series           &         & $-0.59^{***}$ \\
                              &         & $(0.07)$      \\
cpu\_modelatom x5-z8300       &         & $-1.37^{***}$ \\
                              &         & $(0.32)$      \\
cpu\_modelatom x5-z8350       &         & $-1.39^{***}$ \\
                              &         & $(0.19)$      \\
cpu\_modelatom x5-z8550       &         & $-0.92^{***}$ \\
                              &         & $(0.17)$      \\
cpu\_modelatom z8350          &         & $-1.40^{***}$ \\
                              &         & $(0.30)$      \\
cpu\_modelceleron             &         & $-1.01^{***}$ \\
                              &         & $(0.04)$      \\
cpu\_modelcore i3             &         & $-0.47^{***}$ \\
                              &         & $(0.03)$      \\
cpu\_modelcore i5             &         & $-0.13^{***}$ \\
                              &         & $(0.02)$      \\
cpu\_modelcore m              &         & $-0.25^{***}$ \\
                              &         & $(0.07)$      \\
cpu\_modelcortex              &         & $-0.51$       \\
                              &         & $(0.29)$      \\
cpu\_modele-series            &         & $-1.05^{***}$ \\
                              &         & $(0.10)$      \\
cpu\_modelfx                  &         & $-0.47^{**}$  \\
                              &         & $(0.18)$      \\
cpu\_modelpentium             &         & $-0.70^{***}$ \\
                              &         & $(0.05)$      \\
cpu\_modelryzen               &         & $-0.02$       \\
                              &         & $(0.13)$      \\
cpu\_modelxeon                &         & $0.20$        \\
                              &         & $(0.15)$      \\
memory\_typeboth              &         & $0.23^{***}$  \\
                              &         & $(0.05)$      \\
memory\_typehdd               &         & $0.08$        \\
                              &         & $(0.04)$      \\
ssd\_size                     &         & $0.00^{***}$  \\
                              &         & $(0.00)$      \\
hdd\_size                     &         & $-0.00^{**}$  \\
                              &         & $(0.00)$      \\
\hline
R$^2$                         & $0.17$  & $0.85$        \\
Adj. R$^2$                    & $0.16$  & $0.85$        \\
Num. obs.                     & $1042$  & $1042$        \\
RMSE                          & $0.57$  & $$            \\
\hline
\multicolumn{3}{l}{\scriptsize{$^{***}p<0.001$; $^{**}p<0.01$; $^{*}p<0.05$}}
\end{tabular}
\caption{Comparing laptop price models}
\label{table:coefficients}
\end{center}
\end{table}

\hypertarget{standardized-residuals}{%
\subsection{Standardized residuals}\label{standardized-residuals}}

\begin{table}
\begin{center}
\begin{tabular}{l c}
\hline
 & Extended - Multiple regression \\
\hline
(Intercept)                   & $0.00^{***}$  \\
                              & $(0.39)$      \\
companyAcer                   & $-0.04^{**}$  \\
                              & $(0.03)$      \\
companyApple                  & $0.03^{*}$    \\
                              & $(0.07)$      \\
companyAsus                   & $-0.02$       \\
                              & $(0.03)$      \\
companyChuwi                  & $0.02$        \\
                              & $(0.20)$      \\
companyDell                   & $0.02$        \\
                              & $(0.02)$      \\
companyFujitsu                & $-0.01$       \\
                              & $(0.17)$      \\
companyGoogle                 & $0.01$        \\
                              & $(0.15)$      \\
companyHP                     & $0.05^{**}$   \\
                              & $(0.02)$      \\
companyHuawei                 & $0.00$        \\
                              & $(0.18)$      \\
companyLG                     & $0.03^{*}$    \\
                              & $(0.18)$      \\
companyMediacom               & $0.01$        \\
                              & $(0.16)$      \\
companyMicrosoft              & $0.02$        \\
                              & $(0.10)$      \\
companyMSI                    & $0.01$        \\
                              & $(0.05)$      \\
companyRazer                  & $0.03^{*}$    \\
                              & $(0.10)$      \\
companySamsung                & $0.00$        \\
                              & $(0.14)$      \\
companyToshiba                & $0.05^{***}$  \\
                              & $(0.04)$      \\
companyVero                   & $-0.00$       \\
                              & $(0.31)$      \\
companyXiaomi                 & $-0.00$       \\
                              & $(0.14)$      \\
type\_name2 in 1 Convertible  & $0.05^{**}$   \\
                              & $(0.03)$      \\
type\_nameGaming              & $0.12^{***}$  \\
                              & $(0.03)$      \\
type\_nameNetbook             & $-0.01$       \\
                              & $(0.08)$      \\
type\_nameUltrabook           & $0.08^{***}$  \\
                              & $(0.03)$      \\
type\_nameWorkstation         & $0.16^{***}$  \\
                              & $(0.06)$      \\
inches                        & $-0.02$       \\
                              & $(0.02)$      \\
ram                           & $0.19^{***}$  \\
                              & $(0.00)$      \\
screen\_categoryscreen\_big   & $0.10^{***}$  \\
                              & $(0.05)$      \\
screen\_categoryscreen\_small & $0.16^{***}$  \\
                              & $(0.05)$      \\
cpu\_modela10-series          & $-0.05^{***}$ \\
                              & $(0.10)$      \\
cpu\_modela12-series          & $-0.04^{**}$  \\
                              & $(0.10)$      \\
cpu\_modela4-series           & $-0.04^{***}$ \\
                              & $(0.25)$      \\
cpu\_modela6-series           & $-0.10^{***}$ \\
                              & $(0.10)$      \\
cpu\_modela8-series           & $-0.03^{**}$  \\
                              & $(0.25)$      \\
cpu\_modela9-series           & $-0.10^{***}$ \\
                              & $(0.07)$      \\
cpu\_modelatom x5-z8300       & $-0.07^{***}$ \\
                              & $(0.32)$      \\
cpu\_modelatom x5-z8350       & $-0.15^{***}$ \\
                              & $(0.19)$      \\
cpu\_modelatom x5-z8550       & $-0.08^{***}$ \\
                              & $(0.17)$      \\
cpu\_modelatom z8350          & $-0.07^{***}$ \\
                              & $(0.30)$      \\
cpu\_modelceleron             & $-0.42^{***}$ \\
                              & $(0.04)$      \\
cpu\_modelcore i3             & $-0.23^{***}$ \\
                              & $(0.03)$      \\
cpu\_modelcore i5             & $-0.10^{***}$ \\
                              & $(0.02)$      \\
cpu\_modelcore m              & $-0.05^{***}$ \\
                              & $(0.07)$      \\
cpu\_modelcortex              & $-0.03$       \\
                              & $(0.29)$      \\
cpu\_modele-series            & $-0.14^{***}$ \\
                              & $(0.10)$      \\
cpu\_modelfx                  & $-0.03^{**}$  \\
                              & $(0.18)$      \\
cpu\_modelpentium             & $-0.17^{***}$ \\
                              & $(0.05)$      \\
cpu\_modelryzen               & $-0.00$       \\
                              & $(0.13)$      \\
cpu\_modelxeon                & $0.02$        \\
                              & $(0.15)$      \\
memory\_typeboth              & $0.14^{***}$  \\
                              & $(0.05)$      \\
memory\_typehdd               & $0.06$        \\
                              & $(0.04)$      \\
ssd\_size                     & $0.19^{***}$  \\
                              & $(0.00)$      \\
hdd\_size                     & $-0.09^{**}$  \\
                              & $(0.00)$      \\
\hline
R$^2$                         & $0.85$        \\
Adj. R$^2$                    & $0.85$        \\
Sigma                         & $0.24$        \\
Statistic                     & $113.18$      \\
P Value                       & $0.00$        \\
DF                            & $51.00$       \\
Log Likelihood                & $17.43$       \\
AIC                           & $71.14$       \\
BIC                           & $333.43$      \\
Deviance                      & $59.00$       \\
DF Resid.                     & $990$         \\
nobs                          & $1042$        \\
\hline
\multicolumn{2}{l}{\scriptsize{$^{***}p<0.001$; $^{**}p<0.01$; $^{*}p<0.05$}}
\end{tabular}
\caption{Standardised Beta}
\label{table:coefficients}
\end{center}
\end{table}

\hypertarget{y---y-hat-plot}{%
\subsection{Y - Y hat plot}\label{y---y-hat-plot}}

\includegraphics{analysis-pdf_files/figure-latex/unnamed-chunk-24-1.pdf}

We can see on this plot, that the predicted values fit the actual values
fairly well.

\hypertarget{compare-train-and-test-model}{%
\subsection{Compare Train and Test
model}\label{compare-train-and-test-model}}

\begin{table}
\begin{center}
\begin{tabular}{l c c}
\hline
 & Training model & Testing model \\
\hline
(Intercept)                   & $6.64^{***}$  & $6.29^{***}$  \\
                              & $(0.39)$      & $(0.87)$      \\
companyAcer                   & $-0.09^{**}$  & $-0.08$       \\
                              & $(0.03)$      & $(0.07)$      \\
companyApple                  & $0.15^{*}$    & $0.09$        \\
                              & $(0.07)$      & $(0.14)$      \\
companyAsus                   & $-0.04$       & $-0.03$       \\
                              & $(0.03)$      & $(0.06)$      \\
companyChuwi                  & $0.22$        &               \\
                              & $(0.20)$      &               \\
companyDell                   & $0.03$        & $0.10^{*}$    \\
                              & $(0.02)$      & $(0.05)$      \\
companyFujitsu                & $-0.11$       & $0.07$        \\
                              & $(0.17)$      & $(0.24)$      \\
companyGoogle                 & $0.16$        &               \\
                              & $(0.15)$      &               \\
companyHP                     & $0.07^{**}$   & $0.07$        \\
                              & $(0.02)$      & $(0.05)$      \\
companyHuawei                 & $0.01$        &               \\
                              & $(0.18)$      &               \\
companyLG                     & $0.43^{*}$    & $0.10$        \\
                              & $(0.18)$      & $(0.24)$      \\
companyMediacom               & $0.08$        & $-0.30$       \\
                              & $(0.16)$      & $(0.27)$      \\
companyMicrosoft              & $0.18$        &               \\
                              & $(0.10)$      &               \\
companyMSI                    & $0.02$        & $0.16$        \\
                              & $(0.05)$      & $(0.10)$      \\
companyRazer                  & $0.23^{*}$    & $0.21$        \\
                              & $(0.10)$      & $(0.27)$      \\
companySamsung                & $0.04$        & $0.18$        \\
                              & $(0.14)$      & $(0.12)$      \\
companyToshiba                & $0.16^{***}$  & $0.10$        \\
                              & $(0.04)$      & $(0.09)$      \\
companyVero                   & $-0.06$       & $-0.37^{*}$   \\
                              & $(0.31)$      & $(0.18)$      \\
companyXiaomi                 & $-0.03$       & $0.09$        \\
                              & $(0.14)$      & $(0.24)$      \\
type\_name2 in 1 Convertible  & $0.10^{**}$   & $0.16^{**}$   \\
                              & $(0.03)$      & $(0.06)$      \\
type\_nameGaming              & $0.21^{***}$  & $0.16$        \\
                              & $(0.03)$      & $(0.08)$      \\
type\_nameNetbook             & $-0.04$       & $-0.02$       \\
                              & $(0.08)$      & $(0.16)$      \\
type\_nameUltrabook           & $0.14^{***}$  & $0.23^{***}$  \\
                              & $(0.03)$      & $(0.06)$      \\
type\_nameWorkstation         & $0.66^{***}$  & $0.54^{***}$  \\
                              & $(0.06)$      & $(0.11)$      \\
inches                        & $-0.01$       & $0.01$        \\
                              & $(0.02)$      & $(0.05)$      \\
ram                           & $0.02^{***}$  & $0.03^{***}$  \\
                              & $(0.00)$      & $(0.00)$      \\
screen\_categoryscreen\_big   & $0.19^{***}$  & $0.05$        \\
                              & $(0.05)$      & $(0.11)$      \\
screen\_categoryscreen\_small & $0.21^{***}$  & $0.18$        \\
                              & $(0.05)$      & $(0.11)$      \\
cpu\_modela10-series          & $-0.38^{***}$ &               \\
                              & $(0.10)$      &               \\
cpu\_modela12-series          & $-0.32^{**}$  & $-0.38^{*}$   \\
                              & $(0.10)$      & $(0.17)$      \\
cpu\_modela4-series           & $-0.88^{***}$ &               \\
                              & $(0.25)$      &               \\
cpu\_modela6-series           & $-0.78^{***}$ & $-0.77^{***}$ \\
                              & $(0.10)$      & $(0.13)$      \\
cpu\_modela8-series           & $-0.68^{**}$  & $-0.58^{***}$ \\
                              & $(0.25)$      & $(0.15)$      \\
cpu\_modela9-series           & $-0.59^{***}$ & $-0.45^{***}$ \\
                              & $(0.07)$      & $(0.13)$      \\
cpu\_modelatom x5-z8300       & $-1.37^{***}$ &               \\
                              & $(0.32)$      &               \\
cpu\_modelatom x5-z8350       & $-1.39^{***}$ & $-1.26^{***}$ \\
                              & $(0.19)$      & $(0.22)$      \\
cpu\_modelatom x5-z8550       & $-0.92^{***}$ & $-0.44$       \\
                              & $(0.17)$      & $(0.31)$      \\
cpu\_modelatom z8350          & $-1.40^{***}$ &               \\
                              & $(0.30)$      &               \\
cpu\_modelceleron             & $-1.01^{***}$ & $-0.91^{***}$ \\
                              & $(0.04)$      & $(0.10)$      \\
cpu\_modelcore i3             & $-0.47^{***}$ & $-0.47^{***}$ \\
                              & $(0.03)$      & $(0.06)$      \\
cpu\_modelcore i5             & $-0.13^{***}$ & $-0.12^{**}$  \\
                              & $(0.02)$      & $(0.04)$      \\
cpu\_modelcore m              & $-0.25^{***}$ & $-0.05$       \\
                              & $(0.07)$      & $(0.19)$      \\
cpu\_modelcortex              & $-0.51$       &               \\
                              & $(0.29)$      &               \\
cpu\_modele-series            & $-1.05^{***}$ & $-0.91^{***}$ \\
                              & $(0.10)$      & $(0.18)$      \\
cpu\_modelfx                  & $-0.47^{**}$  &               \\
                              & $(0.18)$      &               \\
cpu\_modelpentium             & $-0.70^{***}$ & $-0.50^{***}$ \\
                              & $(0.05)$      & $(0.12)$      \\
cpu\_modelryzen               & $-0.02$       &               \\
                              & $(0.13)$      &               \\
cpu\_modelxeon                & $0.20$        & $0.35$        \\
                              & $(0.15)$      & $(0.27)$      \\
memory\_typeboth              & $0.23^{***}$  & $0.38^{***}$  \\
                              & $(0.05)$      & $(0.11)$      \\
memory\_typehdd               & $0.08$        & $0.17^{*}$    \\
                              & $(0.04)$      & $(0.09)$      \\
ssd\_size                     & $0.00^{***}$  & $0.00^{***}$  \\
                              & $(0.00)$      & $(0.00)$      \\
hdd\_size                     & $-0.00^{**}$  & $-0.00^{**}$  \\
                              & $(0.00)$      & $(0.00)$      \\
\hline
R$^2$                         & $0.85$        & $0.88$        \\
Adj. R$^2$                    & $0.85$        & $0.86$        \\
Num. obs.                     & $1042$        & $261$         \\
\hline
\multicolumn{3}{l}{\scriptsize{$^{***}p<0.001$; $^{**}p<0.01$; $^{*}p<0.05$}}
\end{tabular}
\caption{Modelling laptop prices}
\label{table:coefficients}
\end{center}
\end{table}

\end{document}
